%
\section{Inhaltliche Bestandteile der Seminararbeit}
\label{sec_inhalt}

\subsection{Gliederungspunkte}

Die Seminarausarbeitung sollte {\bf ohne} die Standardseiten wie
\begin{itemize}
\item Titelseite
\item Kurzzusammenfassung
\item Inhaltsverzeichnis
\item Glossar
\item Abk�rzungsverzeichnis
\item Literaturverzeichnis
\item Index (Sachindex)
\end{itemize}
tats�chlich {\bf 20 Seiten} umfassen!
F�r Diplom-/Masterarbeiten gelten entsprechend 60-80 Seiten.

\subsection{Inhalt der Gliederungspunkte}
Unterteilen Sie den eigentlichen Text Ihrer Arbeit in logische, inhaltlich aufeinander aufbauende Gliederungspunkte\index{Gliederungspunkte}.
Stellen Sie sicher, dass der Inhalt jedes Gliederungspunktes auch mit dessen einleitenden S�tzen �bereinstimmt.
Achten Sie bei der Erstellung der einzelnen Gliederungspunkte auf den sprichw�rtlichen "`roten Faden"'\index{roter Faden} , der sich durch die Arbeit ziehen sollte.
Reihen Sie nicht nur einzelne Fakten hintereinander, sondern bringen Sie diese in einen logischen Zusammenhang.
Dies gilt auf allen Gliederungsebenen, d.h.~sowohl f�r den Gesamtaufbau der Arbeit wie auch f�r die einzelnen Unterkapitel.

H�ten Sie sich vor Plagiaten\index{Plagiate}!
Dem Vorwurf des Plagiats setzt man sich auch dann aus, wenn man einer anderen Arbeit zu dicht folgt und seine eigene Arbeit zu sehr an eine andere Arbeit anlehnt.
Die Suchmaschine Google\index{Google} und das WWW bieten einen reichen Schatz an studentischen Arbeiten zu den verschiedensten Themen.
Aber bedenken Sie:
\begin{itemize}
\item Ihr Dozent ist ebenfalls in der Lage, einen Browser zu bedienen.
\item Was Sie im WWW finden, kann auch Ihr Dozent finden.
\item Ihr Dozent hat in der Regel einen besseren �berblick �ber bereits bestehende Arbeiten zum Thema als Sie.
\item Ihr Dozent hat bereits vor Ihnen eine WWW-Recherche zum Thema durchgef�hrt.
\end{itemize}
Grunds�tzlich k�nnen Sie fremde Quellen immer zu Rate ziehen und diese korrekt zitieren.
Eine komplette Arbeit einfach abzuschreiben bringt allerdings auch Ihnen pers�nlich weder einen Erkenntnisgewinn noch Erfahrungen im Erstellen einer wissenschaftlichen Arbeit.


\subsection{Umfang der Gliederungspunkte}
Die einzelnen Unterkapitel sollten entweder selbsterkl�rend sein bzw.~sollte sich deren Zusammenhang aus den bereits vorangegangenen Kapiteln erschlie�en.
Ist dies nicht der Fall, m�ssen Sie eventuell die einzelnen Kapitel umorganisieren bzw. zus�tzliche Erkl�rungen einf�gen.

\subsection{Logischer Zusammenhang}
Generell gilt auch hier: Lesen Sie Ihre Arbeit am Ende komplett in einem St�ck durch.
Wenn Sie glauben, Ihre Arbeit sei logisch konsistent und vollst�ndig, dann lassen Sie diese von einer unbeteiligten Person (am besten einem Nichtfachmann/einer Nichtfachfrau) noch einmal durchlesen.
Diese wird Sie auf eventuell bestehende logische Unzul�nglichkeiten hinweisen.
