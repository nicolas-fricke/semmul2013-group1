\section{Glossary} %Appendix (Glossar)

\begin{description}
\item[Late Fusion:]\index{Late Fusion} combines single-feature clusters by intersecting them. Ensures that all images within a cluster are similar in both features and lead to less or equal to $n/2$ subclusters.

\item[Synset:]\index{Synset} a particular concept which can be expressed by different terms but has one unique identifier. This identifier consists of the word most commonly used to describe the concept, the part of speech, and a number, e.g. drive.v.02.

\item[WordNet:]\index{WordNet} a freely and publicly available ``large lexical database of English nouns, verbs, adjectives and adverbs, grouped into sets of cognitive synonyms (Synsets ), each expressing a distinct concept'' (description on the official web page\footnote{http://wordnet.princeton.edu/})

\item[Markov Clustering Algorithm:]\index{Markov Cluster Algorithm} a graph clustering algorithm for undirected, weighted graphs using random walk to determine clusters

\item[Leacock and Chodorow Similarity:]\index{LCH-Similarity} finds shortest path between two concepts. Uses adapted weights and maximum path length as normalization factors. Is perceived as closer to human understanding than regular path (\cite{budanitsky01} and \cite{pedersen2004wordnet})
\end{description}


\newpage

% Appendix (Akronyme)
\section{Abbreviations and Acronyms}\index{Akronyme}

\begin{tabbing}
\hspace*{3cm}\=  \\ \kill

Bsp \> Beispiel\\
LCH \> Leacock and Chodorow\\
MCL \> Markov Cluster Algorithm\\

\end{tabbing}
