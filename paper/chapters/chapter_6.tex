%
\section{Results Discussion}
\label{sec_discussion}

\todo{Are our results good? Are they biased by something?}
It can generally been said that the quality of the results highly depends on the original image annotations: an inappropriate tag leads the algorithm to ``believe'' that the picture shows something that is actually not present.

\subsection{Testset Quality}
The evaluation results also depend on the test set, which, unfortunately, cannot be clearly right or wrong. Different users will expect different images to be returned according to their defintion of food: When some of the participants of the test set creation were asked which items they considered food, the answers ranged from ``Those that I would like to eat'' to ``Anything that some living organism would eat''. \\
It also has to be assumed that people have different opinions on what images are visually similar, especially since no definition or hints were given to the participants. We used crowdsourcing to deal with these problems and obtain a test set that is supported by the majority of users. So the key question to the quality of the test set is whether \todo{how many users?} participants are enough to obtain a representative result.

\subsection{Image Retrieval}
One of the reasons for the generally poor precision of the image retrieval may lie in poorly annotated images.
Other, more controllable reasons, whatsoever, are to be searched in the Synset detection mechanism. First, the limitation to nouns leads to incorrectly identified Synsets, because adjectives, adverbs and verbs are wrongly matched to nouns if such exist. 
Second, words in other languages than English may be incorrectly matched if they exist in a different meaning in English.\todo{Examples!},
And third, 


\subsection{Semantic Clusters}
MCL based clusters highly depend on quality of keyword clusters. Hard to evaluate, cannot be isolated.\\
Other problems during test set creation include the fact that pictures often contain small or processed items, which makes it hard to identify the exact contents of that picture. The original tags therefore may contain more or contrary information to what the participants could see. \todo{phrase this better or remove}

\subsection{Visual Clusters}
also rather hard to look at in isolation, because method specifically designed for final subclustering. But lack of data for evaluation within subclusters for appropriately sized semantic clusters