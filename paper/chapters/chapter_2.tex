%
\section{Aufbau und Inhalt der wissenschaftlichen Arbeit}
\label{sec_aufbau}

Im vorangegangenen Kapitel hatten wir bereits die Gliederung einer wissenschaftlichen Arbeit im Fachgebiet der Informatik kurz vorgestellt und erl�utert, welche inhaltlichen Punkte in der \glqq Einleitung\grqq\, behandelt werden sollten.
Die folgenden Abschnitte skizzieren inhaltlich die �brigen der bereits genannten Gliederungspunkte.

\subsection{Verwandte Arbeiten und wissenschaftlicher Hintergrund (Related Work)}
%%
Hier sind vor allem zwei inhaltliche Punkte zu ber�cksichtigen:
\begin{itemize}
\item {\bf Notwendige Vorarbeiten und Grundlagen, die zum Verst�ndnis der Arbeit notwendig sind}

Keine bzw. kaum eine Forschungsarbeit beginnt als \glqq tabula rasa\grqq , d.h. meist bauen wir auf  vorhandenen Grundlagen bzw. Vorarbeiten auf.
Die zum Verst�ndnis der eigenen Forschungsarbeit notwendigen Grundlagen\index{Grundlagen} und Voraussetzungen m�ssen in diesem Kapitel skizziert bzw. zusammengefasst werden.
Dabei sollte man vom durchschnittlichen Kenntnisstand eines Informatikers ausgehen, d.h. Allgemeinpl�tze und allzu Grundlegendes hat hier nichts zu suchen.
Genauso soll hier nicht notwendigerweise eine kompletter Wissenschaftszweig in epischer Tiefe ausgebreitet werden, sondern lediglich die zum Verst�ndnis notwendigen Teilbereiche in skizzenhafter Form und mit Angabe von Literaturhinweisen\index{Literaturhinweise} zusammengefasst werden.

\smallskip

\item {\bf Alternative Ans�tze und Forschungsarbeiten zum Thema}

Besonders wichtig ist es, spezielle Vorarbeiten und alternative Forschungsans�tze zum behandelten Thema darzulegen.
Dieser Abschnitt ist der von Ihnen durchgef�hrten Literaturrecherche gewidmet.
Welche Arbeiten zum aktuellen Thema gibt es? Wie sind andere Wissenschaftler an das Thema herangegangen? Hatten Sie Erfolg?

Wichtig ist, dass Sie jede der vorgestellten Arbeiten 
\begin{itemize}
\item korrekt zitieren (Bibliografie),
\item kurz die wichtigsten Ergebnisse bzw. Strategien skizzieren und
\item diese (kurz und knapp) in Zusammenhang mit ihrer eigenen Arbeit stellen. 
\end{itemize}
Wie unterscheidet sich der eigene Ansatz von den vorgestellten Arbeiten? 
Warum ist der eigene Ansatz eventuell erfolgsversprechender? 

\end{itemize}

\subsection{Eigener (wissenschaftlicher) Beitrag}
%%
Hier haben Sie die Freiheit, Ihren eigenen Arbeiten angemessen viel Raum zur Verf�gung zu stellen.
Achten Sie dabei auf einen logischen Aufbau der Darstellung, d.h. Grundlegendes zuerst.
\begin{itemize}
\item Wie sind Sie vorgegangen?
\item Wo gibt es Probleme?
\item Wie werden diese gel�st?
\item Schreiben Sie in verst�ndlicher Weise und dr�cken Sie sich dabei jeweils m�glichst pr�zise, d.h. unmissverst�ndlich aus (vgl. Kap.~\ref{sec_stil})
\item Verwenden Sie Abbildungen, Tabellen und Beispiele.
\item Setzen Sie kein Wissen als implizit vorhanden voraus, sondern sprechen Sie explizit alle Probleme\index{Probleme} und wichtigen Fakten an.
\end{itemize}
Bedenken Sie dabei stets, dass ein Leser nicht dasselbe Wissen besitzen kann wie Sie und das Sie ihm deshalb ihre Ergebnisse erkl�ren m�ssen.


\subsection{Evaluation des (wissenschaftlichen) Beitrags}\index{Evaluation}
%%
Natur- bzw. ingenieurwissenschaftliche Forschung erzielt oft quantitative Ergebnisse, deren Qualit�t objektiv beurteilt werden muss.\index{Qualit�t}
Dies erfolgt �blicherweise mit Hilfe einer speziellen Evaluation, d.h. die Qualit�t der erzielten Ergebnisse muss mit den Ergebnissen anderer Arbeiten auf objektive Weise verglichen werden k�nnen.

\begin{itemize}
\item Oft existieren zu diesem Zweck Benchmarks\index{Benchmark}, die aber auch selbst, angepasst an die eigene, spezielle Problemstellung zusammengestellt werden k�nnen.
\item Wird ein Benchmark bzw. ein Evaluationsverfahren selbst erstellt, sollten die Autoren diesen �ffentlich zur Verf�gung stellen, damit die erzielten Ergebnisse nachvollziehbar werden.

{\bf Merke: Was man nicht nachvollziehen kann, wird angezweifelt.}
\item Wird ein existierender Benchmark verwendet, muss dieser korrekt zitiert werden und eigene Ergebnisse mit bereits bekannten Ergebnissen in Relation gestellt werden.
\end{itemize}



\subsection{Diskussion der Evaluationsergebnisse}
%%
Wurde eine Evaluation der erzielten Ergebnisse durchgef�hrt, m�ssen diese diskutiert werden.
Dabei sollten (falls jeweils zutreffend) folgende Fragen beantwortet werden:\index{Evaluation}
\begin{itemize}
\item Warum ist das eigene Ergebnis besser/schlechter als das zum Vergleich herangezogene?
\item Sind die erzielten Ergebnisse objektiv oder gibt es Gr�nde, diese eventuell in Zweifel zu ziehen?
\item Welche Vorbedingungen m�ssten ver�ndert werden, um bessere Ergebnisse bzw. eine objektivere Evaluation zu erzielen?
\item Falls die Evaluation f�r unseren speziellen Fall nicht aussagekr�ftig (genug) ist, wie sollte diese ver�ndert werden?
\end{itemize}


\subsection{Zusammenfassung und Ausblick}\index{Zusammenfassung}

In diesem Abschnitt sollten die erzielten Ergebnisse noch einmal kurz zusammengefasst werden und ein Ausblick auf weiterf�hrende Forschungs- und Entwicklungsarbeiten gegeben werden (vgl. Kap.~\ref{sec_conclusion}).