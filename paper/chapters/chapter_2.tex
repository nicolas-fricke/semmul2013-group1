%
\section{Related Work}
\label{sec_relatedwork}

Much research been done recently in image clustering and semantic clustering, with application areas in image segmentation, compact representation of large image sets, search space reduction and avoiding the semantic gap in content based image retrieval ($http://www.cscjournals.org/csc/manuscript/Journals/IJIP/Finalversion/Camera_ready_IJIP-304.pdf$) \\
However, most of this work presents new algorithms for one of the above use cases, not methods to retrieve training data

Related Subjects: Image Annotation, semantic clustering, content-based image retrieval

\subsection{Semantic Clustering and Tags}
%% http://ilpubs.stanford.edu:8090/890/2/cluster-wsdm09.pdf: clustered search results based on tags, but for web pages
Current issues with tag-based search and clustering, are related to the lack of a defined tag vocabulary (e.g. the use of synonyms, homonyms, variations in spelling etc.), and elaborated on more closely in \cite{Auer2011}

\subsection{Image Annotation and Content-Based Image Retrieval}
%% ?? http://ganges.usc.edu/pgroupW/images/6/6b/Cvm2012.pdf: detect objects and organize images based on the relations of the objects within.
%% http://ieeexplore.ieee.org/xpls/abs_all.jsp?arnumber=718510: review of visual features for cbir
%% http://hal.archives-ouvertes.fr/docs/00/54/85/85/PDF/cvpr06_lana.pdf: success of pyramidal approach in refining clusters

Ideas exist to use visual features to semantically analyze and classify images. \cite{Liu2007} and \cite{Zhang2012} provide good summaries and evaluations of the different approaches how this could be done. Both conclude that this so-called \emph{Automatic Image Annotation} \index{Automatic Image Annotation} is computation-intensive and not yet fully mature.

\subsection{Approaches to Combine Semantics and Visuals}
%% http://infolab.stanford.edu/~wangz/project/imsearch/review/MTA/neela.pdf: summary of semantic image interpretation; Overview of foundations 
%% http://link.springer.com/content/pdf/10.1007 2Fs11042-008-0247-7.pdf: annotating images based on visuals and "folksonomies" (existing tags); Alternative Approach
