% Vierte Seite = Hier geht's eigentlich richtig los
\section{Retrieving Images in Clusters}
\label{sec_introduction}


\subsection{Problem Statement \& Motivation}
training data for image categorization and content detection \\
flickr and other online photo communities are good sources for annotated images \\
problems: low annotation quality, only search for specific term (with different meanings and visual characteristics) \\
for example, want to test the quality of my algorithm and identifying different foods: would have to think of all kinds of food and search and filter images manually

\bigskip

\emph{What do we do?} \\
clustering: creating homogeneous groups of semantically and visually similar pictures \\

\emph{Why do we do that?} \\
seminar challenge: cluster 1 million pictures of the MIR1M flickr file set \\
improving the complex task of searching for pictures according to a given keyword \\
facing different challenges like: multiple meanings of the keyword, bad picture annotations, taking semantic and visual information of a picture into account \\

\subsection{Clustered Tree Nodes Approach}
idea: provide ready-to-use semantically and visually homogeneous image clusters for a given topic. Span tree of subordinate pictures, retrieve related images and cluster them to distinguish different settings of the pictures and to identify outliers.

\bigskip

After giving an overview of Related Work in chapter 2, we will present our methods for retrieving (chapter 3) and clustering (chapter 4) appropriate images. Chapter 5 explains how we evaluate our approach, while the evaluation results will be discussed in chapter 6. At last, chapter 7 gives ideas for improvement and possible future work.
