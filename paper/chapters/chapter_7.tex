%
\section{Conclusion and Future Work}
\label{sec_future}

\todo{How to improve, what other approaches to take}
\begin{itemize}
\item Was haben wir erreicht?
\item Was sind die naechsten Schritte?
\item Wie koennen die gewonnenen Ergebnisse angewendet werden?
\item Warum sind die gewonnenen Ergebnisse bedeutend?
\end{itemize}

Within this work we described a new approach to cluster images in homogeneous groups with the help of the knowledge base WordNet. Our goal was to assign every image a semantic meaning based on their meta information retrieved from the MIRFLICKR-1M file set and matching these to their meaning based on Synsets from WordNet. \\
We build a tool which creates a search tree containing hyponyms and meronyms for a given search term and groups matching images into semantically and visually similar clusters. 

\bigskip
With that approach we are the first, \\
its a big topic, it creates no training data but a differentiated result \\
Build a good basement for future steps \\

During our evaluation we found out, that the basic idea still shows some problems concerning satisfying results but we already detect possible steps to work to increase the general outcom
e. 

\subsection{Semantic}
use more or other WordNet relations\\
improve keyword clusters by re-clustering large clusters\\
better synset detection (still see faulty recognition of tags), use groups and albums additionally, description with named-entity recognition \\
use different approach apart from WordNet as a knowledge base like DBpedia \footnote{http://dbpedia.org/About}

\subsection{Visuals}

use of high level features: would it improve the implemented visual clustering?
