% Muster f�r die Seminarausarbeitung
% HPI Potsdam

\documentclass[11pt, a4paper]{article}

\usepackage{ngerman}
\usepackage[latin1]{inputenc} %Korrekte Kodierung der Umlaute nach ISO 8859-1
\usepackage[T1]{fontenc} %Korrekte Kodierung der Umlaute nach ISO 8859-1
\usepackage{makeidx} %Zur automatischen Indexerstellung
\usepackage{amsfonts}
\usepackage{amssymb}
\usepackage{epsfig}   % Zum Einbinden von Bildern
\usepackage{url}      % Korrekter Satz von URLs
\usepackage{color}    % Verwendung von Farben
\usepackage{listings} % Korrekter Satz von Listings und Quellcode


%Hilfs-Fonts - ohne Serifen (hier f�r Tabellen)
\newfont{\bib}{cmss8 scaled 1040}
\newfont{\bibf}{cmssbx8 scaled 1040}

\definecolor{lightgray}{gray}{0.85}

%Seitenformat-Definitionen
\topmargin0mm
\textwidth147mm
\textheight214mm
\evensidemargin5mm
\oddsidemargin5mm
\footskip19mm
\parindent=0in

\makeindex % legt das Index-File an


\begin{document}          

\begin{titlepage}
  \begin{center} 
    \mbox{}
    \vspace{1cm}
    
    {\huge Titel der Seminararbeit \\[1em] {\LARGE ggf.~mit Untertitel}}  
        
    \vspace{5cm}
    
    Seminararbeit im Seminar \\[1em]
    {\large \sc Titel des Seminars} \\[1em]
    Sommersemester 2013 \\[1em]
    Hasso-Plattner-Institut f�r Softwaresystemtechnik GmbH \\[1em]
    Universit�t Potsdam
    
    \vspace{4cm}
    
		vorgelegt von
		
    \vspace{1em}
    
		{\Large Mandy Roick} \\
		{\Large Claudia Exeler} \\
		{\Large Tino Junge} \\
		{\Large Nicolas Fricke}
		
    \vspace{4em}
    
    30.~August 2013
  \end{center}
\end{titlepage}


\setcounter{page}{1}

% Zweite Seite = Kurzzusammenfassung
\begin{center}
{\bf Kurzzusammenfassung} 
\end{center}

\noindent
An dieser Stelle erfolgt eine knappe Zusammenfassung der vorliegenden Arbeit ([engl.] Abstract)\index{Abstract}, die maximal ca.~200 Worte umfassen sollte. 
Der Sinn und Zweck dieser Kurzzusammenfassung liegt darin, einem interessierten Leser die Entscheidung zu erleichtern, die vorliegende Arbeit �berhaupt zu lesen bzw.~vor dem Lesen der Arbeit erst einmal in Erfahrung zu bringen, worum es geht.
Also eine knappe, motivierende Hinf�hrung zum Problem und wie sie es gel�st haben.

\bigskip

Wenn Sie eine Kurzzusammenfassung schreiben, bedenken Sie, dass diese oft auch alleine publiziert wird, d.h. sie sollte unabh�ngig vom nachfolgend explizit dargestellten Inhalt der Arbeit f�r den Leser verst�ndlich sein.
Daher ist es immer sinnvoll, diese Zusammenfassung erst ganz am Ende zu schreiben, wenn Sie die eigentliche Arbeit bereits abgeschlossen haben.

\newpage

% Dritte Seite = Inhaltsverzeichnis
\tableofcontents 

\newpage

% Vierte Seite = Hier geht's eigentlich richtig los
\section{Retrieving Images in Clusters}
\label{sec_introduction}


\subsection{Problem Statement}

What do we do? \\

clustering: creating homogeneous	groups of semantically and visually similar pictures \\


Why do we do that? \\


seminar challenge: cluster 1M pictures of the MIR1M flickr file set \\



\subsection{Motivation}

 
\newpage
\section{Related Work}
\label{sec_relatedwork}

Much research has been done recently in image clustering and semantic clustering, with application areas in image segmentation, compact representation of large image sets, search space reduction and avoiding the semantic gap in content based image retrieval \cite{Lim2011}. \\
However, most of this work focuses on new algorithms for one of the above use cases, not on methods to generate training data. \\

One algorithm for retrieving training data for image analysis is presented in \cite{Orendovici2010}. The algorithm collects training data for computational analysis of the quality of photographs from Flickr. But, instead of analyzing pictures automatically according to their tags and comments, the paper presents a tool for collecting user votes. Comments were only used to retrieve terms for describing image quality.

\bigskip

Some of the closely related subjects like Semantic Clustering and Content-Based Image Retrieval are presented in this chapter.

\subsection{Semantic Clustering and Tags}
The idea of clustering search results based on tags and other annotations has been implemented before by \cite{Ramage2009} but for web pages instead of images. The main difference is that documents such as web pages consist of words, so their content itself can be used for semantic analysis.
Current issues with tag-based search and clustering are mostly related to the lack of a defined tag vocabulary (e.g. the use of synonyms, homonyms, variations in spelling etc.), and elaborated on more closely in \cite{Auer2011}.

\bigskip

There are already some approaches using WordNet for finding semantic similarities between different words, like \cite{richardson1994using}. They motivate using a knowledge base like WordNet to deal with general problems of the natural language. This means facing different meanings of a word and strong relations between different words. With the help of a hierarchical concept graph (HCG), consisting of hyponymns and meronyms for a specific entity, the semantic similarity between different words can be determined. In conclusion, the main challenge is to have a very accurate sense disambiguator which automatically and correctly assign words to their WordNet meaning. The existence of such a tool is indispensable for all classification approaches and they analyzed the WordNet semantic tagger as very promising.

Another approach is using a WordNet-based clustering technique for text documents, which is described in \cite{sedding2004wordnet}.  They tried to solve the problem of synonyms and ambiguity of words within texts, while adding a part-of-speech tag to every word based on knowledge provided by WordNet. Unfortunately the found out, that including synonyms and hypernyms does not improve the effectiveness of clustering, which they mainly relate to noise which comes with incorrect word interpretation of WordNet. 

\todo{More semantic related word?}

\subsection{Image Annotation and Content-Based Image Retrieval}
%% ?? http://ganges.usc.edu/pgroupW/images/6/6b/Cvm2012.pdf: detect objects and organize images based on the relations of the objects within.
%% http://ieeexplore.ieee.org/xpls/abs_all.jsp?arnumber=718510: review of visual features for cbir

Ideas exist to use visual features to semantically analyze and classify images. \cite{Liu2007} and \cite{Zhang2012} provide good summaries and evaluations of the different approaches how this could be done. Both conclude that this so-called \emph{Automatic Image Annotation} \index{Automatic Image Annotation} is computation-intensive and not yet fully mature.

\todo{Include more Related Work for Image Annotation}
%% http://infolab.stanford.edu/~wangz/project/imsearch/review/MTA/neela.pdf: summary of semantic image interpretation; Overview of foundations
\bigskip
One approach that combines semantics and visuals to analyze pictures in a so called \emph{visual folksonomy} is \cite{Lindstaedt2009}. The idea is trying to annotate images, with a controlled vocabulary, based on visual features and existing tags. Their goal, however, is to create additional annotations for not or poorly tagged images. \\
Another approach is presented by \cite{cai2004hierarchical} with the aim to cluster images returned by a WWW image search. In contrast to pictures from a folksonomy, image search results are connected to a web page with context and link information which is used by the algorithm to cluster images.

\newpage
%
\section{Image Tree Based on Wordnet}
\label{sec_wordnetsearchtree}

\subsection{Wordnet}
Related to the offical web page WordNet is described as a freely and publicly available "large lexical database of english nouns, verbs, adjectives and adverbs, grouped into sets of cognitive synonyms (synsets), each expressing a distinct concept." (ref: http://wordnet.princeton.edu/). Synsets are semantically linked with each other e.g. hyponyms and meronyms relations. 

network of synsets for discovering semantics between words \\
  

\subsection{Constructing a Searchtree}
two typical types of queries: more or less generic object descriptors, and places \\
actually construct multiple searchtrees if more than one synset found for a searchterm (i.e. train coach and motorized vehicle for car)\\
use hyponym relation to span tree of specializations (i.e. apple, banana for fruit; bus, sportscar for car)\\
if no hyponyms (usually the case for geographic terms), use part-meronyms\\

\subsection{Synset Detection}
for each tag and word in title, try to find synset (limiting ourselves to nouns, because they are usually the ones describing the depicted concepts). Further source options: description (named entity recognition necessary), comments (noticed little relation to picture), group and album names (for both preprocessing needed to match any to wordnet)  \\
problem: multiple possible synsets for a word, how to find correct meaning? \\
use best-first search (with limited queue for complexity reasons. idea is that paths at more than position x are unlikely to become best candidate anyways) \\
still erroneous with words that are meant in a way that is unknown to WordNet, i.e. canon as the camera model is interpreted as [definition of canon.n.01]  \\
therefore preprocessing removes all tags that include numbers. Blacklist could filter even more but would also filter canon in its real sense, and generally not desirable to be flexible with respect to the tag vocabulary.  \\
also removes special characters (more likely to be found on WordNet, and more likely to be identical with other unmatchable tags) 

\subsection{Assigning Picture to Synsets}
for higher recall: find strongly co-occurring tags that could not be mapped to synset \\
strong co-occurrence defined on tf-idf (else camera models would be strong co-occurrence with many synsets)
observed that it is useful to find translations etc. but of course also introduces noise \\
take all pictures that are annotated with at least one of the related tags or the synset itself.
\newpage
%
\section{Inhaltliche Bestandteile der Seminararbeit}
\label{sec_inhalt}

\subsection{Gliederungspunkte}

Die Seminarausarbeitung sollte {\bf ohne} die Standardseiten wie
\begin{itemize}
\item Titelseite
\item Kurzzusammenfassung
\item Inhaltsverzeichnis
\item Glossar
\item Abk�rzungsverzeichnis
\item Literaturverzeichnis
\item Index (Sachindex)
\end{itemize}
tats�chlich {\bf 20 Seiten} umfassen!
F�r Diplom-/Masterarbeiten gelten entsprechend 60-80 Seiten.

\subsection{Inhalt der Gliederungspunkte}
Unterteilen Sie den eigentlichen Text Ihrer Arbeit in logische, inhaltlich aufeinander aufbauende Gliederungspunkte\index{Gliederungspunkte}.
Stellen Sie sicher, dass der Inhalt jedes Gliederungspunktes auch mit dessen einleitenden S�tzen �bereinstimmt.
Achten Sie bei der Erstellung der einzelnen Gliederungspunkte auf den sprichw�rtlichen "`roten Faden"'\index{roter Faden} , der sich durch die Arbeit ziehen sollte.
Reihen Sie nicht nur einzelne Fakten hintereinander, sondern bringen Sie diese in einen logischen Zusammenhang.
Dies gilt auf allen Gliederungsebenen, d.h.~sowohl f�r den Gesamtaufbau der Arbeit wie auch f�r die einzelnen Unterkapitel.

H�ten Sie sich vor Plagiaten\index{Plagiate}!
Dem Vorwurf des Plagiats setzt man sich auch dann aus, wenn man einer anderen Arbeit zu dicht folgt und seine eigene Arbeit zu sehr an eine andere Arbeit anlehnt.
Die Suchmaschine Google\index{Google} und das WWW bieten einen reichen Schatz an studentischen Arbeiten zu den verschiedensten Themen.
Aber bedenken Sie:
\begin{itemize}
\item Ihr Dozent ist ebenfalls in der Lage, einen Browser zu bedienen.
\item Was Sie im WWW finden, kann auch Ihr Dozent finden.
\item Ihr Dozent hat in der Regel einen besseren �berblick �ber bereits bestehende Arbeiten zum Thema als Sie.
\item Ihr Dozent hat bereits vor Ihnen eine WWW-Recherche zum Thema durchgef�hrt.
\end{itemize}
Grunds�tzlich k�nnen Sie fremde Quellen immer zu Rate ziehen und diese korrekt zitieren.
Eine komplette Arbeit einfach abzuschreiben bringt allerdings auch Ihnen pers�nlich weder einen Erkenntnisgewinn noch Erfahrungen im Erstellen einer wissenschaftlichen Arbeit.


\subsection{Umfang der Gliederungspunkte}
Die einzelnen Unterkapitel sollten entweder selbsterkl�rend sein bzw.~sollte sich deren Zusammenhang aus den bereits vorangegangenen Kapiteln erschlie�en.
Ist dies nicht der Fall, m�ssen Sie eventuell die einzelnen Kapitel umorganisieren bzw. zus�tzliche Erkl�rungen einf�gen.

\subsection{Logischer Zusammenhang}
Generell gilt auch hier: Lesen Sie Ihre Arbeit am Ende komplett in einem St�ck durch.
Wenn Sie glauben, Ihre Arbeit sei logisch konsistent und vollst�ndig, dann lassen Sie diese von einer unbeteiligten Person (am besten einem Nichtfachmann/einer Nichtfachfrau) noch einmal durchlesen.
Diese wird Sie auf eventuell bestehende logische Unzul�nglichkeiten hinweisen.

\newpage
%
\section{Evaluation}
\label{sec_evaluation}

We evaluated our tool on a set of 9,201 images, which are a subset of the 1 million images of the MIRFLICKR-1M\footnote{http://press.liacs.nl/mirflickr/} file set, and the query term ``food''. Since no comparable algorithms exist, the evaluation is mainly aimed at obtaining the best values for the parameters and at providing a basis for comparison of further improvements and future work.

\subsection{Test set}
\label{sec_testset}
No gold standard is available to tell us which pictures show food and how similar the images are. The creation of such standards and training data is exactly the task we want to facilitate with this work.\\ 
To test the quality of our algorithm, we wrote a tool allowing us a crowdsourced generation of the needed reference data by the general public. This was achieved in two phases:\\

First, the users were shown random picture out of the 9,201 test set images and asked whether it shows food or not. We normalized these answers, so that there is only one vote per user per picture. In the case a user rated a picture multiple times, the value is determined by the ratio of positive (\emph{``shows food''}) and negative (\emph{``does not show food''}) votes of each user on one picture. We consider all those images as showing food that received at least 50\% positive votes. With over 35,000 clicks by more than 20 participants, 1,142 images out of the total 9,201 images were identified to show food. \\

We then also required data on the semantic and visual similarity of these pictures. Therefore, in the second phase, the users were shown pairs of images, on which we knew from phase 1 that they contain food, and asked to compare them. They could choose between three levels of semantic similarity: \emph{not similar}, \emph{same object}, and \emph{same object and same context}, and two levels of visual similarity: \emph{similar}, and \emph{not similar}.\\
Among the 12,962 votes of more than 30 participants were 757 pairs of images with same objects, 345 pairs with same object and same context, as well as 1,854 pairs of visually similar images. Multiple votes on one pair were rare (39 cases), and therefore simply not taken into account if they contradicted each other.

\subsection{Quality Indicators}
The evaluation focuses on the following four main aspects of our algorithm:
\begin{enumerate}
\item Retrieval of matching images
\item Semantic hierarchy and clusters
\item Visual clustering
\end{enumerate}

We measure the quality of the image retrieval (\emph{1.}) by calculating the F-Measure for the returned pictures, comparing our algorithm's result to the crowdsourced generated test set of phase 1. \todo{should we compare synset detection mechanisms?} 

The quality of the hierarchy of the retrieved images (\emph{2.}) is based on the \emph{same object} and \emph{same object and context} pairs: The  minimal path distance for an annotated pair of pictures can be calculated and used to determine the closeness of two images $closeness(x,y) = 1/distance(x,y)$. Averaging this value over all pairs of a similarity category returns a value between 0 and 1 (below referenced as $c_o$ for same object pairs, $c_c$ for same object and same context pairs, and $c_n$ for not similar pairs), with the optimal values being 1 for positive (similar) pairs, and 0 for negative (non-similar) pairs. 
We include the keyword clustering in this evaluation by handling the clusters in a node as its children. Consequentially, the perfect score of 1 can only be reached when two semantically similar images are not only in the same node but also in the same semantic subcluster.

Visual similarity (\emph{3.}) is evaluated on the whole test set, because not enough comparison data is available to get valuable results if only comparisons within semantic clusters were used. Once again, F-Measure is used as indicator.

\todo{vary parameters given by frontend, trying to find best configuration}

\subsection{Results}
\label{sec_results}

\subsubsection*{Image retrieval}

Our image retrieval has a precision of p = 50.2\% and recall of r = 85.9\% on the ``food'' query, before execution of the semantic clustering that removes outliers. Without the use of co-occurring tags described in section \ref{sec_picturestonodes}, both values show no significant difference with p = 50.5\% and r = 85.4\%.
After the semantic clustering, the measures depend on the \emph{minimal mcl cluster size} parameter. The results for different values of this parameter are presented in table \ref{tab_retrievalevaluation}.\\

\begin{table}[h]
   \begin{tabular}{| p{2.2cm}| p{2.2cm}| p{2cm} || p{2cm} | p{2cm} | p{2cm} |}
    \hline
    \emph{mcl clustering threshold} & \emph{minimal mcl cluster size} & \emph{minimal node size} & \emph{precision} & \emph{recall} & \emph{f-measure} \\ \hline
    0 	& 0 	& 0 & 0.501532 & 0.859143 & 0.633344 \\ \hline
    0 	& 5 	& 0 & 0.559668 & 0.783036 & 0.652773 \\ \hline
    5 	& 5 	& 5 & 0.549815 & 0.798214 & 0.651129 \\ \hline     
    15 	& 25 &  5 & 0.615894 & 0.747321 & 0.675272 \\ \hline
    15 	& 10 & 15 & 0.585333 & 0.783929 & 0.670229 \\ \hline
    15 	& 25 & 15 & 0.695298 & 0.672791 & 0.683859 \\ \hline
    	100 	& 100 & 100 & 0.757858 & 0.569554 & 0.650350 \\ \hline
    \end{tabular}
    \caption{Precision and recall of the image retrieval}
	\label{tab_retrievalevaluation}
\end{table}

\begin{table}[h]
    \begin{tabular}{| p{2cm}| p{2cm}| p{2cm} || p{2cm} | p{2cm} | p{2cm} |}
    \hline
    2	& 6	& 2	& 0.575916 & 0.769904 & 0.658929 \\ \hline    
    4	& 2	& 2	& 0.518719 & 0.836395 & 0.640322 \\ \hline	      
    4	& 2	& 4	& 0.520458 & 0.834646 & 0.641129 \\ \hline	
    4	& 6	& 2	& 0.562267 & 0.790026 & 0.656966 \\ \hline     
    4 	& 4 	& 4 & 0.550663 & 0.798775 & 0.651910 \\ \hline
 	4	& 6	& 4	& 0.568766 & 0.774278 & 0.655798 \\ \hline    
    6 	& 6 	& 6 & 0.567619 & 0.782152 & 0.657837 \\ \hline  
	5 	& 10 & 5 & 0.612903 & 0.748031 & 0.673759 \\ \hline
    5 	& 15 & 5 & 0.644391 & 0.708661 & 0.675000 \\ \hline  
   	5 	& 10 & 10 & 0.598080 & 0.762905 & 0.670511 \\ \hline 
    10 	& 10 & 10 & 0.598080 & 0.762905 & 0.670511 \\ \hline 
    10 	& 10 & 20 & 0.578165 & 0.783027 & 0.665180 \\ \hline    
    10 	& 20 & 20 & 0.642631 & 0.709536 & 0.674428 \\ \hline
    10 	& 15 & 10 & 0.631539 & 0.728784 & 0.676686 \\ \hline  
    10 	& 20 & 10 & 0.666102 & 0.687664 & 0.676711 \\ \hline
    15 	& 25 & 10 & 0.655201 & 0.714286 & 0.683469 \\ \hline
    15 	& 20 & 15 & 0.661716 & 0.701662 & 0.681104 \\ \hline    
    20 	& 20 & 20 & 0.642631 & 0.709536 & 0.674428 \\ \hline
    20 	& 25 & 20 & 0.677951 & 0.683290 & 0.680610 \\ \hline  
    50	& 50 & 50 & 0.728135 & 0.604549 & 0.660612 \\ \hline
    100 	& 100 & 100 & 0.757858 & 0.569554 & 0.650350 \\ \hline
    \end{tabular}
    \caption{Additional precision and recall of the image retrieval}
\end{table}


\begin{table}[h]
    \begin{tabular}{| p{1.8cm} | p{1.8cm} || p{1.5cm} | p{1.5cm} | p{1.5cm} |}
    \hline
    ?	& \emph{minimal node size} & $c_o $ & $c_c$ & $c_n$ \\ \hline
    value 1 		& value 1 	& value 1 	& value 1	& value 1 \\ \hline
    value 2 		& value 2	& value 2 	& value 2	& value 2 \\ \hline
    value 3	 	& value 3	& value 3 	& value 3	& value 3 \\
    \hline
    \end{tabular}
    \caption{Semantic quality measures}
	\label{tab_treeevaluation}
\end{table}

\subsubsection*{Semantic and visual clustering}

Within our second phase we let users decide which of the pictures declared as showing food are semantically and visually similar. 

Image tuples voted with same object: 757 \\
Image tuples voted with same object and same context: 1102 \\
Image tuples voted as visual similar: 1854 \\

After retrieving our evaluation votes from the users, we compared the result of our tool with one from the evaluation. Therefore we calculated the distance between every two images, which are declared as similar or not, within our calculated search tree. \todo{Picture of a search tree, which illustrates the evaluation script concerning how the distance is calculated?}

Average distance for same object  is 2.77091633466  \\
Average distance for same object and same context is 2.64246575342 \\
Average distance for not similar  is 2.87943462898 \\



\newpage
%
\section{Results Discussion}
\label{sec_discussion}

It can generally been said that the quality of the results highly depends on the original image annotations: an inappropriate tag leads the algorithm to ``believe'' that the picture shows something that is actually not present.

\subsection{Test Set Quality}
The evaluation results also depend on the test set, which, unfortunately, cannot be clearly right or wrong. Different users will expect different images to be returned according to their definition of food: When some of the participants of the test set creation were asked which items they considered food, the answers ranged from ``Those that I would like to eat'' to ``Anything that some living organism would eat''. \\
Another problem lies in the fact that pictures often contain small or processed items, which makes it hard to identify the exact contents of that picture. That's why during test set creation participants could probably not see the described content or interpret the image different in contrast to the original tags of the image. 
It also has to be assumed that people have different opinions on what images are visually similar, especially since no definition or hints were given to the participants. We used crowdsourcing to deal with these problems and obtain a test set that is supported by the majority of users. So the key question to the quality of the test set is whether there are enough participants to obtain a representative result.

\subsection{Image Retrieval}
One of the reasons for the generally poor precision of the image retrieval may lie in poorly annotated images.
Other, more controllable reasons, whatsoever, are to be searched in the Synset detection mechanism.
First, the limitation to nouns leads to incorrectly identified Synsets because adjectives, adverbs and verbs are wrongly matched to nouns if such exist, e.g. \emph{fall} as a verb for falling under the influence of gravity and as a noun for autumn. 
Second, words in other languages than English may be incorrectly matched if they exist in a different meaning in English, e.g. \emph{gift} for present in English and poison in German.
And third, the assumption that tags on the same picture are semantically close does not hold in all cases (like words with different meanings also known as homographs), e.g. a \emph{cherry} on a plate made of \emph{wood} would be assigned to the cherry tree meaning instead of cherry as a fruit. \\
The first two reasons can be addressed by using a more sophisticated ontology and similarity measure, but the third requires a rethinking of the Synset detection algorithm.

\bigskip
Key points on MCL: generally useful (removes more non-food images than food images, else recall would go down with precision staying the same), interestingly best results when minimal mcl cluster size larger than minimal node size (i.e. small nodes are deleted anyways)

\subsection{Semantic Clusters}
closeness confirms idea that semantically close images are closer to each other in the tree than semantically different ones
however, all values seem rather close to average tree distance. average is also what mostly varies with mcl parameters
same objects should at least be in same node -> closeness >= 0.5; same context also in same cluster -> closeness 1.0
major reason: testset. some  users only participated in second phase, did not know all images contained food already (so voted very different kinds of food as same). but also for others, since no definition of 'same object' given, difficult choice of level of abstraction / granularity (are cake and cupcake the same? cake and cookie?). participants later stated insecurity about these votes and own inconsistency when no pairs appeared they considered same.\\
unfortunately for tool, best parameters for recall/precision do not correlate with best parameter values for semantic similarity 
\bigskip
Our semantic evaluation method is a good method to get an quality measure of the complete semantic hierarchy. However, this makes it hard to make statements about the individual semantic clustering steps. For example, the results of the keyword based clustering highly depend on the quality of the calculated keyword clusters. Those are hard to judge in isolation, though.\\	


\subsection{Visual Clusters}
also rather hard to look at in isolation, because method specifically designed for final subclustering. But lack of data for evaluation within subclusters for appropriately sized semantic clusters



%% Anh�nge
\newpage
\begin{appendix}


\section{Glossary} %Appendix (Glossar)

\begin{description}
\item[Folksonomy:]\index{Folksonomy} a collaboratively created content classification system derived from annotating and categorizing content with tags by users

\item[Late Fusion:]\index{Late Fusion} combines single-feature clusters by intersecting them. Ensures that all images within a cluster are similar in both features and lead to less or equal to $n/2$ subclusters.

\item[Hyponym:]\index{Hyponym} semantic \emph{is-a} relation, describes a partial aspect of a superior term. The superior term is called ``hypernym''

\item[Markov Clustering Algorithm:]\index{Markov Cluster Algorithm} a graph clustering algorithm for undirected, weighted graphs using random walk to determine clusters

\item[Meronym:]\index{Meronym} describes a \emph{is-part-of} or \emph{is-member-of} relation. The superior holonym consists of multiple meronyms, e.g. a ``finger'' is a meronym of the ``hand''

\item[Leacock and Chodorow Similarity:]\index{LCH-Similarity} finds shortest path between two concepts. Uses adapted weights and maximum path length as normalization factors. Is perceived as closer to human understanding than regular path (\cite{budanitsky01} and \cite{pedersen2004wordnet})

\item[Ontology:]\index{Ontology} ``an explicit, formal specification of a shared conceptualization'' \cite{gruber1993translation}
%formally ordered representation of a set of terminologies and their connections within a certain scope

\item[Synset:]\index{Synset} a particular concept which can be expressed by different terms but has one unique identifier. This identifier consists of the word most commonly used to describe the concept, the part of speech, and a number, e.g. drive.v.02.

\item[WordNet:]\index{WordNet} a freely and publicly available ``large lexical database of English nouns, verbs, adjectives and adverbs, grouped into sets of cognitive synonyms (Synsets ), each expressing a distinct concept'' (description on the official web page\footnote{http://wordnet.princeton.edu/})

\end{description}


\newpage

% Appendix (Akronyme)
\section{Abbreviations and Acronyms}

\begin{tabbing}
\hspace*{3cm}\=  \\ \kill

HSV \> Hue, Saturation, Value\\
LCH \> Leacock and Chodorow\\
MCL \> Markov Cluster Algorithm\\
tf-idf \> Term frequency and inverse document frequency\\

\end{tabbing}


\end{appendix}


%Hier kommt das Literaturverzeichnis
\newpage

\addcontentsline{toc}{section}{Literaturverzeichnis} % Zeile f�r das Inhaltsverzeichnis

\bibliography{bibfile}
\bibliographystyle{alphadin}

\newpage

%Hierhin kommt der Index (Sachverzeichnis)
\addcontentsline{toc}{section}{Index} % Dies ist die Zeile f�r das Inhaltsverzeichnis
\flushbottom                                    
\printindex

\end{document}
